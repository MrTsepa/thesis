\documentclass[12pt,a4paper]{extarticle}

\usepackage[top=20mm, bottom=20mm, left=30mm, right=20mm]{geometry}
\usepackage[T1]{fontenc}
\usepackage[utf8x]{inputenc}
\usepackage[russian]{babel}
\usepackage{amsmath}
\usepackage{amssymb}


\newcommand{\R}{\mathbb{R}}
\newcommand{\E}{\mathcal{E}}
\newcommand{\encoder}{\operatorname{enc}}
\newcommand{\decoder}{\operatorname{dec}}

\title{Анализ проблемы вложения графов и применимости к ней нового метода -- структурной функции потерь}

\begin{document}

    \begin{titlepage}
        \pagenumbering{gobble}
        \clearpage
        \pagestyle{empty}

        \begin{center}
            \textrm{Министерство образования и науки Российской Федерации}
            \\[5mm]
            Федеральное государственное автономное образовательное учреждение\\
            высшего профессионального образования\\
            <<Московский физико-технический институт\\
            {(государственный университет)}>>\\[5mm]
            Факультет радиотехники и кибернетики\\[5mm]
            Кафедра проблем передачи информации и анализа данных\\[50mm]
            \textbf{АНАЛИЗ ПРОБЛЕМЫ ВЛОЖЕНИЯ ГРАФОВ И ПРИМЕНИМОСТИ К НЕЙ НОВОГО МЕТОДА -- СТРУКТУРНОЙ ФУНКЦИИ ПОТЕРЬ}
            \\[8mm]
            \textrm{Выпускная квалификационная работа}
            \\
            (бакалаврская работа)\\[7mm]

            Направление подготовки: 03.03.01 Прикладные математика и физика\\[40mm]
        \end{center}

        \noindent Выполнил:\\
        студент 411а группы \hspace{0.9cm} \underline{\hspace{4cm}} Цепа Станислав Евгеньевич\\[2mm]

        \noindent Научный руководитель:\\
        к.ф.-м.н. \hspace{3.1cm} \underline{\hspace{4cm}} Панов Максим Евгеньевич
        \\[20mm]

        \begin{center}
            Москва 2017
        \end{center}

    \end{titlepage}

    \pagenumbering{arabic}
    \setcounter{page}{2}

    \tableofcontents

    \newpage

    \section{Введение}
    Большинство алгоритмов машинного обучения основано на использовании
    признаков, являющихся числами (а также иногда категориями).
    Однако, далеко не все сущности, которые хотелось бы использовать в качестве признаков,
    можно легко перевести в числовой вид.
    Примерами таких объектов являются слова в тексте или вершины в графе.
    В нашей работе мы рассматривали задачу вложения: как каждой отдельной сущности (в нашем
    случае вершине графа) сопоставить точку вещественного $n$-мерного пространства,
    или, проще говоря, $n$ вещественных чисел, чтобы относительное
    расположение этих точек лучшим образом отображало структуру графа.
    Мы проверяли качество вложения, проводя TODO классификацию и кластеризацию
    стандартными методами и вычисляя соответствующие метрики.
    Стоить отметить, что вложение является задачей обучения без учителя,
    то есть при построении не учитывает известные метки классов, но только
    переводит вершины в числовые признаки.
    После этого полученные числовые признаки могут быть использованы для любых целей, в том числе и для
    применения в алгоритмах классификации и кластеризации.

    \section{Постановка задачи}
    В данном разделе мы формализуем задачу вложения графов и демонстрируем как можно разделить ее на подзадачи.

    Пусть дан граф $G = (V, E)$, его матрица смежности $A \in \R^{n \times n}$.
    Алгоритм, строящий вложение графа по этим данным, обычно состоит из следующих частей:
    \begin{itemize}
        \item функция кодирования (\textit{encoder}) строит вложение графа
            \[\encoder: V \to \R^{|V| \times d}.\]
            далее будем обозначать вложение $\E_i = \encoder(v_i)$
        \item функция декодирования (\textit{decoder}) оценивает схожесть двух элементов вложения,
            в идеальном случае $\decoder(\E_i, \E_j) = A_{ij}$
            \[\decoder: \R^{d} \times \R^{d} \to \R\]
        \item функция потерь (\textit{loss}) оценивает насколько хорошо функции декодирования удалось
            восстановить исходную матрицу.
    \end{itemize}
    
    Далее будем называть эту функцию кодировщик(\textit{encoder}).


    \subsection{}



    \bibliography{main}
    \bibliographystyle{plain}

\end{document}
